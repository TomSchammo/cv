%----------------------------------------------------------------------------------------
%	DOKUMENTDEFINITION
%----------------------------------------------------------------------------------------

% !TeX ltex: language=de

% article Klasse, da wir die Seite vollständig anpassen und keine Lebenslaufvorlage verwenden möchten
\documentclass[a4paper,11pt]{article}

%----------------------------------------------------------------------------------------
%	SCHRIFTART
%----------------------------------------------------------------------------------------

% % fontspec erlaubt die direkte Verwendung von TTF/OTF-Schriften
% \usepackage{fontspec}
% \defaultfontfeatures{Ligatures=TeX}

% % angepasst für ShareLaTeX
% \setmainfont[
% SmallCapsFont = Fontin-SmallCaps.otf,
% BoldFont = Fontin-Bold.otf,
% ItalicFont = Fontin-Italic.otf
% ]
% {Fontin.otf}

%----------------------------------------------------------------------------------------
%	PAKETE
%----------------------------------------------------------------------------------------
\usepackage{url}
\usepackage{parskip}

%weitere Pakete zur Formatierung
\RequirePackage{color}
\RequirePackage{graphicx}
\usepackage[usenames,dvipsnames]{xcolor}
\usepackage[scale=0.9]{geometry}

%tabularx Umgebung
\usepackage{tabularx}

%für Listen innerhalb des Erfahrungsbereichs
\usepackage{enumitem}

% zentrierte Version des 'X' Spaltentyps
\newcolumntype{C}{>{\centering\arraybackslash}X}

%um das Überlaufen von tabular auf die nächsten Seiten zu verhindern
\usepackage{supertabular}
\newlength{\fullcollw}
\setlength{\fullcollw}{0.47\textwidth}

%benutzerdefinierte \section
\usepackage{titlesec}
\usepackage{multicol}
\usepackage{multirow}

%Lebenslaufabschnitte inspiriert von:
%http://stefano.italians.nl/archives/26
\titleformat{\section}{\Large\scshape\raggedright}{}{0em}{}[\titlerule]
\titlespacing{\section}{0pt}{10pt}{10pt}

%für Publikationen
\usepackage[style=authoryear,sorting=ynt, maxbibnames=2]{biblatex}

%Setup des hyperref Pakets und Farben für Links
\usepackage[unicode, draft=false]{hyperref}
\definecolor{linkcolour}{rgb}{0,0.2,0.6}
\hypersetup{colorlinks,breaklinks,urlcolor=linkcolour,linkcolor=linkcolour}
\addbibresource{citations.bib}
\setlength\bibitemsep{1em}

%für soziale Symbole
\usepackage{fontawesome5}

%Rahmen zur Fehlerbehebung der Seite
%\usepackage{showframe}

%----------------------------------------------------------------------------------------
%	BEGINN DOKUMENT
%----------------------------------------------------------------------------------------
\begin{document}

% nicht nummerierte Seiten
\pagestyle{empty}

%----------------------------------------------------------------------------------------
%	TITEL
%----------------------------------------------------------------------------------------

% \begin{tabularx}{\linewidth}{ @{}X X@{} }
% \huge{Dein Name}\vspace{2pt} & \hfill \emoji{incoming-envelope} email@email.com \\
% \raisebox{-0.05\height}\faGithub\ Benutzername \ | \
% \raisebox{-0.00\height}\faLinkedin\ Benutzername \ | \ \raisebox{-0.05\height}\faGlobe \ meineSeite.com  & \hfill \emoji{calling} Telefonnummer
% \end{tabularx}

\begin{tabularx}{\linewidth}{@{} C @{}}
\Huge{Tom Schammo} \\[7.5pt]
\href{https://github.com/TomSchammo}{\raisebox{-0.05\height}\faGithub\ Tom Schammo} \ $|$ \
\href{https://www.linkedin.com/in/tom-schammo-a1496a308/}{\raisebox{-0.05\height}\faLinkedin\ Tom Schammo} \ $|$ \
% \href{https://mysite.com}{\raisebox{-0.05\height}\faGlobe \ meineSeite.com} \ $|$ \
\href{mailto:tom@tomschammo.com}{\raisebox{-0.05\height}\faEnvelope \ tom@tomschammo.com} \ $|$ \
\href{tel:+352 621 48 0040}{\raisebox{-0.05\height}\faMobile \ +352 621 48 0040} \\
\end{tabularx}

%----------------------------------------------------------------------------------------
%	AUSBILDUNG
%----------------------------------------------------------------------------------------
\section{Ausbildung}
\begin{tabularx}{\linewidth}{@{}l X@{}}
    2019 - 2024 & B.Sc. (Informatik) an der \textbf{Universität Tübingen} \hfill \normalsize (Note: 2,36) \\[3.75pt]
\multicolumn{2}{@{}X@{}}{
\begin{minipage}[t]{\linewidth}
    \footnotesize{
    Kurse umfassen:
    \begin{itemize}[nosep,after=\strut, leftmargin=1em, itemsep=3pt]
        \item[--] Programming Ultra Low Power Architectures: Ein M.Sc. Kurs, der Studierenden Stromspaarzustände, Hardware-Treiber, Busse und Hardware-Interrupts näher bringt.
        \item[--] Massively Parallel Computing: Ein M.Sc. Kurs, in dem die Programmierung von GPUs mittels CUDA gelehrt wird, um diverse Probleme effizienter zu lösen, wie z.B. Matrixmultiplikation oder die Manipulation von Datenstrukturen.
        \item[--] Teamprojekt: Ein Kurs, in dem Studierende lernen ein Software-Projekt im Team zu planen und umzusetzen. Insbesondere wird die Versionskontrolle mittels git erlernt.
        \item[--] Abschlussarbeit: Das Ziel meiner Abschlussarbeit war die (Audio-)Schlüsselworterkennung auf eingebetteten Geräten zu verbessern.    \end{itemize}} \end{minipage}
}
\end{tabularx}

%----------------------------------------------------------------------------------------
% BERUFSERFAHRUNG
%----------------------------------------------------------------------------------------

%Interessen/ Schlüsselwörter/ Zusammenfassung
% \section{Zusammenfassung}
% Dieser Lebenslauf kann auch automatisch mithilfe von GitHub Actions erstellt und veröffentlicht werden. Für Details, \href{https://github.com/jitinnair1/autoCV}{hier klicken}.

%Erfahrung
\section{Berufserfahrung}

\begin{tabularx}{\linewidth}{ @{}l r@{} }
\textbf{Universität Tübingen} & \hfill Nov 2023 - heute \\
    \footnotesize{Wissenschaftlicher Mitarbeiter}\\[3.75pt]
\multicolumn{2}{@{}X@{}}{
    \footnotesize{
    Im Rahmen eines Forschungsprojekts zu selbstfahrenden Autos habe ich die vollständige Entwicklung einer \href{Lidar-Punktwolken-Augmentierungsbibliothek}{https://github.com/ekut-es/LidarAug} übernommen.
    Diese stellt verschiedene SOTA-Augmentierungsmethoden durch ein einfach zu bedienendes Python-Frontend bereit, während die Leistung aufgrund des effizienten C++-Backends erheblich verbessert wird.

    Als alleiniger Entwickler war ich für die konzeptionelle Planung, Programmierung und Testung der
    Bibliothek verantwortlich, wodurch ich meine Fähigkeiten entwickelte, mittelgroße Projekte eigenständig von Anfang bis Ende durchzuführen.
}}  \\
\end{tabularx}

\begin{tabularx}{\linewidth}{ @{}l r@{} }

    \textbf{Octoshrew Ltd.} &  \hfill Nov 2022 - Nov 2023 \\
    \footnotesize{Junior Entwickler}\\[3.75pt]
\multicolumn{2}{@{}X@{}}{
    \footnotesize{
        Als Junior Entwickler war ich an der Entwicklung modernster Pfadplanungs- und autonomen Steuerungssysteme
        für Drohnen beteiligt. Insbesondere war ich für die Pfadfindung verantwortlich und entwickelte die Datenstrukturen und Algorithmen, um
        eine sichere und effiziente Navigation im 3D-Raum zu gewährleisten, wodurch ich meine Fähigkeiten in Algorithmen und Leistungsoptimierung/Parallelisierung verbesserte.
    }}
\end{tabularx}

%Projekte
\section{Projekte}

\begin{tabularx}{\linewidth}{ @{}l r@{} }
\textbf{Implementierung und Analyse verschiedener Eingabequellen auf der Ultratrail-Architektur} \\[3.75pt]
\multicolumn{2}{@{}X@{}}{
    \footnotesize{
    Für meine Abschlussarbeit arbeitete ich an der Implementierung eines Rust-Treibers für den Ultratrail AI-Beschleuniger auf dem PULPissimo-Board
    (ein Entwicklungsboard für einen von der ETH Zürich entwickelten RISC-V-Chip), um eine effiziente Schlüsselworterkennung in Audio zu ermöglichen.

    Durch die Verbesserung der Sicherheit mittels der Programmiersprache Rust gegenüber des bestehenden C-Treibers
    verbesserte ich meine Fähigkeiten und erweiterte mein Wissen in der eingebetteten Programmierung sowie in der Fehlersuche auf Hardwareebene
    mithilfe von Werkzeugen wie Oszilloskopen oder Logikanalysatoren.}
}  \\

\textbf{Programming Ultra Low Power Architectures} \\[3.75pt]
\multicolumn{2}{@{}X@{}}{

    \footnotesize{
        Ich baute einen 'Wecker' also proof-of-concept mit einem Entwicklungsboard und Peripheriegeräten (LED-Display, Tastatur, Lautsprecher).
        Der Wecker nutzte Tiefschlafzustände, um sehr energieeffizient zu sein und überlebte 12 Stunden allein auf einem kleinen Kondensator.
    Zum Vergleich: Mit einer AA-Batterie würde der Wecker 130'000 bis 220'000 Stunden im Energiesparmodus halten.

    Das Projekt bot mir die Möglichkeit, Einblicke in die Nutzung von Hardwaredokumentationen sowie in die
    Entwicklung und Fehlersuche von Hardwaretreibern und eingebetteten Anwendungen zu gewinnen.
    Zusätzlich bot es die Gelegenheit, die Nutzung von Prozessor-Schlafzuständen und Werkzeugen zur
    Messung des Stromverbrauchs zu erkunden, was zu verbesserten batteriebetriebenen Designs führte.
}}\\
\end{tabularx}

%----------------------------------------------------------------------------------------
%	PUBLIKATIONEN
%----------------------------------------------------------------------------------------
% \section{Publikationen}
% \begin{refsection}[citations.bib]
% \nocite{*}
% \printbibliography[heading=none]
% \end{refsection}

%----------------------------------------------------------------------------------------
%	FÄHIGKEITEN
%----------------------------------------------------------------------------------------
\section{Fähigkeiten}
\footnotesize{
\begin{tabularx}{\linewidth}{@{}l X@{}}
    \textbf{Sprachen}: & { Deutsch (Muttersprache), Luxemburgisch (Muttersprache), Englisch (C2), Französisch (B2)} \\
    \textbf{Programmiersprachen}: & { C, C++, CUDA, Python, Rust, RISC-V Assembler, Mips Assembler} \\
    \textbf{Entwicklungstools}: & { Make, CMake, Git, GitHub, GitLab, Linux, Docker, Logic Analyzer, Oszilloskop} \\
    \textbf{Frameworks}: & { Libtorch, Boost, Google Test, ROS, PyTest, Numpy} \\
 % \textbf{Soft Skills}{: } \\
 % \textbf{Interessensgebiete}{: } \\
\end{tabularx}
}
% \vfill
% \center{\footnotesize Zuletzt aktualisiert: \today}
\end{document}
